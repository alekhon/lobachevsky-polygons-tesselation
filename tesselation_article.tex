\\documentclass{article}
\\usepackage[utf8]{inputenc}
\\usepackage[english, russian]{babel}
\\usepackage{graphicx}
\\usepackage[left=2.8cm, right=2.8cm, top=4cm, bottom=3cm]{geometry}
\\usepackage{fancyhdr}
\\usepackage[usenames]{color}
\\usepackage{colortbl}
\\usepackage{amssymb}
\\usepackage{amsmath}
\\usepackage{svg}

\\pagestyle{fancy}
\\fancyhf{}
\\rhead{\\thepage}
\\renewcommand{\\headrulewidth}{0pt}
\\usepackage[colorlinks=true, allcolors=blue]{hyperref}

\\title{Замощение плоскости Лобачевского}
\\author{Алехин Герман}

\\begin{document}

\\maketitle

\\section{Введение}
\\par Геометрия Лобачевского, как известно, имеет ряд различий с Евклидовой геометрией. Одним из таких различий является возможности замощения плоскости Лобачевского равными правильными многоугольниками. 
\\par В Евклидовой геометрии таких замощений всего 3: правильными треугольниками, квадратами и правильными шестиугольниками. Это связано с тем, что в Евклидовой геометрии сумма углов треугольника равна $\\pi$, и поэтому углы правильных многогранников однозначно определены, а для замощения необходимо чтобы угол многоугольника делил $2\\pi$ нацело, поскольку иначе в одной вершине не будет сходиться целое число многоугольников. 
\\par В геометрии Лобачевского же сумма углов треугольника может принимать любое значение на интервале $(0; \\pi)$. Тогда сумма углов правильного $n-$угольника может принимать любое значение на интервале $(0; (n - 2)\\pi)$, а значение одного его угла - соответственно на интервале $(0; \\frac{(n - 2)\\pi}{n})$ поэтому для любого числа сторон правильного многоугольника найдутся размеры его углов, делящие $2\\pi$ нацело.  Значит, существуют замощения плоскости Лобачевского правильными многоугольниками с любым числом сторон. Моя задача заключается в алгоритмическом частичном построении на компьютере такого замощения по трем натуральным числам: $P$ - число сторон многоугольника, $Q$ - число многоугольников, сходящихся в одной вершине, а также число слоев замощения. Такое замощение на плоскости Лобачевского будет существовать при $(P-2)(Q-2) > 4$ (подробнее об этом сказано в статье Дугласа Данэма $Hyperbolic$ $Symmetry$, 1986 год)

\\section{Основные определения и факты}
\\subsection{Модель Пуанкаре геометрии Лобачевского в круге}
\\par \\textit{Модель Пуанкаре геометрии Лобачевского в круге} - это модель, в которой точками плосоксти являются точки внутренности единичного круга на комплексной плоскости, его граница ($\\textit{абсолют}$) считается бесконечно удаленной, прямыми являются диаметры абсолюта и дуги окружностей, ортогональных абсолюту, а движениями являтся дробно-линейные отображения комплексной плоскости, сохраняющие множество точек единичного круга. Такая модель конформна, то есть величины углов между прямыми с точки зрения геометрии Лобачевского совпадают с Евклидовыми углами между соответствующими им дугами окружностей. Расстояния между точками $a$ и $b$ в такой модели вычисляются как $|{ln[a, b; \\alpha, \\beta]}|$, где $\\alpha$ и $\\beta$ - точки пересечения окружности, содержащей прямую, соответствующую прямой, проходящей через $a$ и $b$ с абсолютом, a  $[a, b; \\alpha, \\beta] = \\frac{a - \\alpha}{b - \\alpha}:\\frac{a - \\beta}{b - \\beta}$ - двойное отношение точек $a, b, \\alpha, \\beta$ на комплексной плоскости. Именно в этой модели будет выполнено окончательное построение многоугольников.
\\subsection{Модель геометрии Лобачевского на гиперболоиде}
\\par \\textit{Модель геометрии Лобачевского на гиперболоиде} - это модель, в которой точками плосоксти являются точки верхней чашки двуполостного гиперболоида $x^2 + y^2 - z^2 = -1$. Прямые в этой модели - это кривые пересечения верхней чашки гиперболоида с плоскостями, проходящими через начало координат. Движения в этой модели - это линейные преобразования $\\mathbb{R}^3$, относительно которых гиперболоид инвариантен, мы будем пользоваться такими из них, как Евклидов поворот относительно оси $Oz$ и \\textit{гиперболический поворот относительно оси $Oy$}. Модель на гипрболоиде будет использоваться для реализации преобразований, поскольку все используемые преобразования, как будет показано далее, раскладываются на композиции указанных Евклидовых и гиперболических поворотов, которые имеют очень удобные для программирования матричные представления.

\\par \\textit{Гиперболический поворот относительно оси Oy} - преобразование, задающееся матрицей вида:

\\begin{center}

\\begin{bmatrix}
    ch\\psi & 0 & sh\\psi \\\\
    0 & 1 & 0 \\\\
    sh\\psi & 0 & ch\\psi
\\end{bmatrix}
\\end{center}
где $\\psi$ - аргумент гиперболического поворота. Геометрически $\\psi$ - это удвоенная
площадь криволинейного треугольника, ограниченного лучами из начала координат в точку на кривой пересечения гиперболоида с плоскостью $xOz$ и в ее образ при гиперболическом повороте и частью этой кривой между этими двумя точками. Такое преобразование сохраняет наш гиперболоид.
\\subsection{Изометрия}

\\par Интересна \\textit{изометрия} между этими двумя моделями. Оказывается, она представляет из себя центральную проекцию с центром в точке $(0, 0, -1)$ верхней чашки гиперболоида на плосоксть $xOy$ (подразумевается, что эта плоскость - комплексная, $0$ находится в точке $(0, 0, 0)$, $1$ - в точке $(1, 0, 0)$, $i$ - в точке $(0, 1, 0)$). Такое преобразование переведет чашку гиперболоида во внутренность единичного круга, прямые в модели на гиперболоиде - в прямые в модели Пуанкаре, при этом движения будут соответствовать движениям.

\\section{Алгоритм}

\\subsection{Преобразования}
Построение замощения начинается с построения в модели Пуанкаре правильного $P-$угольника с углом $\\frac{2\\pi}{Q}$ при вершине и центром симметрии в точке $0$ (назовем его \\textit{первым многоугольником}). Алгоритм его построения будет описан далее. Все остальные многоугольники будут строиться из исходного композициями 
следующих преобразований:

\\begin{itemize}
    \\item $a$ - поворот относительно одной конкретной вершины первого многоугольника на $\\frac{2\\pi}{Q}$ (эту вершину будем называть \\textit{нулевой вершиной};
    \\item $b$ - поворот относительно середины стороны первого многоугольника, содержащей центр поворота $a$ и лежащей от него по часовой стрелке, на $\\pi$.
\\end{itemize}
Эти движения я обозначил буквами, поэтому их композиции будут образовывать \\textit{слова}. Слова я в дальнейшем буду использовать для идентификации построенных многоугольников. При этом слова, в отличие от композиций, я буду записывать слева направо, к примеру слово $"aaab"$ означает, что сначала было трижды выполнено движение $a$, а потом движение $b$. Так как оба движения $a$ и $b$ первого рода, то и все слова будут обозначать движения первого рода. Говоря что многоугольник или вершина \\textit{строится по слову}, я буду иметь в виду то, что слово переводит в этот многоугольник или вершину первый многоугольник или нулевую вершину соответственно. Словом многоугольника или вершины я буду называть слово, по которому этот многоугольник или вершина строится.
\\par Так как движения будут реализованы в модели на гиперболоиде, то первым шагом в выполнении каждого из них станет центральная проекция из $(0, 0, -1)$ на гиперболоид (изометрия).
Заметим, что Евклидов поворот на некоторый угол относительно точки $0$ соответствует повороту относительно той же точки на тот же угол. Как мы знаем, любой поворот можно получить, сопрягая поворот вокруг точки $0$ с движением, переводящим из нужный нам центр в $0$. Поэтому указанные повороты ($a$ и $b$) на гиперболоиде мы будем выполнять следующим образом: 
\\begin{enumerate}
\\item Сделаем поворот $R_{Oz}^\\phi$ на такой угол $\\phi$, чтобы центр выполняего преобразования (нулевая вершина или середина стороны для $a$ или $b$ соответственно) оказался в плоскости $xOz$;
\\item Сделаем гиперболический поворот $H_{Oy}^{-arsh(x)}$, где $x$ - абсцисса образа центра при $R_{Oz}^\\phi$ из пункта $1$ (почему аргумент имеет такой вид, можно понять, рассмотрев сечение гиперболоида плоскостью $xOz$ как гиперболу $z^2 - x^2 = 1$ и воспользовавшись геометрическим смыслом ареасинуса), переводящий образ центра при $R_{Oz}^\\phi$ из пункта $1$ в точку $(0, 0, 1)$, соответствующую центру круга. Таким образом мы получили движение $H_{Oy}^{-arsh(x)} \\circ R_{Oz}^\\phi$, переводящее центр выполняемого преобразования в $(0, 0, 1)$.
\\item Сделаем поворот $R_{Oz}^\\frac{2\\pi}{Q}$ или на $R_{Oz}^\\pi$ (для выполнения преобразование $a$ или $b$ соответственно);
\\item Сделаем композицию $R_{Oz}^{-\\phi} {\\circ} H_{Oy}^{arsh(x)}$, обратную композиции $H_{Oy}^{-arsh(x)} \\circ R_{Oz}^\\phi$
\\end{enumerate}
После этого выполним проекцию с центром в $(0, 0, -1)$ на плоскость $xOy$, чтобы снова перейти к модели в круге.
\\\\
\\par Таким образом, пользуясь изометрией и свойствами движений, мы научились выполнять движения $a$ и $b$ через переход к гиперболоиду.

\\subsection{Перебор слов}
\\par Следующей задачей является поиск алгоритма, перебирающего слова и, соответственно, строящего многоугольники так, чтобы в итоге было построено заданное число слоев замощения, и при этом ни один многоугольник не был построен более одного раза. 
\\par Заметим, что слово $"ab"$ переводит первый многоугольник в себя и циклически сдвигает его вершины на $1$ по часовой стрелке. Поскольку $"ab"$ движение, то оно сохраняет расстояния, поэтому центр круга (он же центр первого многоугольника - единственная точка, равноудаленная от всех его вершин) останется на месте. Таким образом $"ab"$ - движение первого рода с неподвижной точкой, при этом на каждую вершину первого многоугольника оно действует как поворот относительно этой самой неподвижной точки на $\\frac{-2\\pi}{P}$. Тогда $"ab"$ - и есть поворот на $\\frac{-2\\pi}{P}$ относительно центра круга. Запомним этот факт. Кроме всего прочего, он позволяет, построив одну вершину первого многоугольника, построить и все остальные.
\\par В связи с этим удобно ввести нумерацию вершин первого многоугольника от $0$ до $P - 1$ по часовой стрелке, начиная с нулевой вершины. Договоримся каждый раз применять преобразование к первому многоугольнику, обходя вершины в порядке их нумерации и присваивая каждой построенной вершине номер ее прообраза. Такое правило задает нумерацию вершин для всех построенных многоугольников, при этом, зная слово некоторого многоугольника, мы автоматически знаем слова всех его вершин (действительно, словом вершины с номером $k$ будет слово самого многоугольника, к которому в начале добавили $"ab"$ $k$ раз). Этим механизмом нумерации я буду пользоваться при построении.
\\par Центральным понятием моего алгоритма замощения является \\textit{флаг} - построение из нескольких многоугольников, полученных последовательными поворотами одного конкретного многоугольника относительно одной выбранной его вершины на $\\frac{2\\pi}{Q}$ (будем говорить, что эта вершина \\textit{строит флаг} или что \\textit{флаг строится на} этой вершине). Количество поворотов зависит от типа флага: флаги \\textit{$1-$го типа} состоят из $Q - 2$ многоугольников, флаги \\textit{$2-$го типа} - из $Q - 3$ многоугольников, а флаги \\textit{$3-$го типа} - из $Q - 4$ многоугольников. Отмечу, что флаг $3-$го типа будет применяться только при $P=3$, более подробно это будет рассмотрено далее. Примеры флагов можно видеть на \\textit{Рис.1} и \\textit{Рис.2}. На рисунках многоугольники, образующие флаг, выделены зеленым; многоугольник, вершина которого строит флаг, выделен синим; сама вершина, строящая флаг, выделена синей окружностью; первый многоугольник (если он не был выделен синим) выделен красным:

\\begin{figure} [h!]
    \\centering
    \\renewcommand{\\figurename}{Рис. 1}
    \\renewcommand{\\thefigure}{}
    \\center{\\includegraphics[scale=0.3]{3_8_1.png} \\includegraphics[scale=0.3]{3_8_2.png} \\includegraphics[scale=0.3]{3_8_3.png}}
    \\caption{Флаги $1$-го, $2$-го и $3-$го типов на нулевой веришине для $P=3$ и $Q=8$.}%пояснение к рисунку%}
    \\label{fig:5_5_1.png}
\\end{figure}

\\begin{figure} [h!]
    \\centering
    \\renewcommand{\\figurename}{Рис. 2}
    \\renewcommand{\\thefigure}{}
    \\center{\\includegraphics[scale=0.3]{6_4_1.png} \\includegraphics[scale=0.3]{6_4_2.png}}
    \\caption{Флаги $1$-го и $2$-го типов на веришине с номером $2$ многоугольника, строящегося по слову $"aab"$ для $P=6$ и $Q=4$.}
    \\label{fig:6_4_1.png}
\\end{figure}

\\par Флаг на вершине с номером $k$ многоугольника со словом $W$ можно построить следующим образом:
\\begin{enumerate}
    \\item Построим флаг нужного типа на нулевой вершине (он строится последовательными преобразованиями $a$ исходного многоугольника);
    \\item Применим $"ab"$ $k$ раз. Тогда нулевая вершина перейдет в вершину первого многоугольника с номером $k$, а флаг - во флаг такого же типа, построенный на ней;
    \\item Применим преобразование, соответствующее слову $W$. Тогда вершина первого многоугольника с номером $k$ перейдет в вершину с таким же номером, но уже в многоугольнике, строящемуся по слову $W$, а флаг - во флаг того же типа, построенный на ней;
\\end{enumerate}

\\begin{figure} [h!]
    \\centering
    \\renewcommand{\\figurename}{Рис. 3}
    \\renewcommand{\\thefigure}{}
    \\center{\\includegraphics[scale=0.25]{4_1.png} \\includegraphics[scale=0.25]{4_2.png} \\includegraphics[scale=0.25]{4_3.png}}
    \\caption{Этапы построения флага $1-$го типа на веришине с номером $1$ многоугольника, строящегося по слову $"a"$ для $P=4$ и $Q=5$.}
    \\label{fig:4_1.png}
\\end{figure}
Ранее я упоминал понятие \\textit{слой замощения}, но не давал ему четкого определения. Определим его рекуррентно: $L_{1}$ - множество, всех многоугольников без учета нумерации вершин, имеющих общую вершину с первым многоугольником и не совпадающие с ним, $\\forall n \\in \\mathbb{N}\\setminus \\{1\\} $ $L_{n}$ - множество всех многоугольников без учета нумерации вершин, имеющих общую вершину с каким-нибудь многоугольником из $L_{n-1}$, при этом не совпадающих ни с одним многоугольником из $L_{k}$ $\\forall k \\in \\mathbb{N}, k \\leq n$. Назовем \\textit{внешними} вершинами $L_{n}$ те вершины его многоугольников, которые не совпадают ни с одной вершиной многоугольников из $L_{n-1}$. Тогда заметим, что флаг, построенный на внешней вершине $L_{n}$ и не содержащий многоугольников из $L_{n}$ целиком лежит в $L_{n+1}$. Будем строить флаги на внешних вершинах $L_{n}$ следующим образом: 
\\begin{itemize}
    \\item Если вершина не является общей ни для каких двух многоугольников $L_{n-1}$ , то построим на ней флаг $1-$го типа;
    \\item Если вершина является общей для двух многоугольников $L_{n}$, то построим на ней флаг $2-$го типа, расматривая ее как вершину того из сходящихся в ней многоугольников, который находится дальше против часовой стрелки (тот, который переводится в оставшийся поворотом на $\\frac{-2\\pi}{Q}$ относительно вершины, строящей флаг).
    \\item Если вершина является общей для трех многоугольников $L_{n}$ (это возможно только при P = 3, в этом случае внешняя вершина треугольника, имеющего общую сторону с треугольником из $L_{n-1}$, принадлежит еще двум многоугольникам из $L_{n}$), то построим на ней флаг $3-$го типа, расматривая ее как вершину того из сходящихся в ней многоугольников, который находится дальше против часовой стрелки (тот, который переводится в оставшиеся поворотами на $\\frac{-2\\pi}{Q}$ и $\\frac{-4\\pi}{Q}$ на  относительно вершины, строящей флаг).
\\end{itemize}
Заметим, что каждый построенный таким образом флаг вместе с многоугольниками из $L_{n}$, сходящимися в вершине, на которой он строится, составляют ровно $Q - 1$ многоугольников, сходящихся в этой вершине. Оставшийся $1$ многоугольник войдет во флаг, построенный на следующей против часовой стрелке вершине (среди всех многоугольников каждого флага только один имеет общую вершину с многоугольником из $L_{n}$, отличную от той, на которой этот флаг строится). Поэтому флаги, построенные по такому алгоритму на внешних вершинах $L_{n}$ полностью образуют $L_{n + 1}$, не покрывая никакой многоугольник дважды!

\\begin{figure} [h!]
    \\centering
    \\renewcommand{\\figurename}{Рис. 4}
    \\renewcommand{\\thefigure}{}
    \\center{\\includegraphics[scale=0.2]{f_3_1.png} $\\to$ \\includegraphics[scale=0.2]{f_3_2.png} $\\to$ \\includegraphics[scale=0.2]{f_3_3.png}}
    \\caption{Построение флагов на внешних вершинах слоя $L_{1}$ для $P = 3$ и $Q = 7$}
    \\label{fig:4_1.png}
\\end{figure}


\\begin{figure} [h!]
    \\centering
    \\renewcommand{\\figurename}{Рис. 5}
    \\renewcommand{\\thefigure}{}
    \\center{\\includegraphics[scale=0.25]{f_7_1.png} $\\to$ \\includegraphics[scale=0.25]{f_7_2.png} $\\to$ \\includegraphics[scale=0.25]{f_7_3.png} $\\to$ \\includegraphics[scale=0.25]{f_7_4.png} $\\to$ \\includegraphics[scale=0.25]{f_7_5.png}}
    \\caption{Построение флагов на одном из многоугольников слоя $L_{1}$ для $P = 7$ и $Q = 4$}
    \\label{fig:4_1.png}
\\end{figure}

\\subsection{Построения}
Осталось описать непосредственно способ построения изображения полученного замощения. Фактически оно разбивается на 4 этапа:
\\begin{enumerate}
    \\item Вычисление координат нулевой вершины и центра поворота $b$;
    \\item Вычисление координат наборов вершин всех интересующих многоугольников;
    \\item Построение сторон всех многоугольников.
\\end{enumerate}
Заметим, что если выполнен пункт $1$, то при помощи описанных преобразований можно выполнить пункт $2$. Поэтому фактически остаются $2$ задачи: вычислить координаты нулевой вершины и центра поворота $b$ по данным $P$ и $Q$ и научиться строить отрезок прямой в геометрии Лобачевского по его концам.
\\subsubsection{Нулевая вершина и центр поворота {$b$}}
Для удобства и определенности нулевую вершину расположим на положительной части оси ординат. Тогда ее координаты будут $(0, \\rho_e )$, где $\\rho_e$ - Евклидово расстояние от центра круга до вершины первого многоугольника. Тогда координаты центра поворота $b$ будут $(h_esin(\\frac{\\pi}{P}), h_ecos(\\frac{\\pi}{P}))$, где $h_e$ - Евклидово расстояние от центра круга до середины стороны первого многоугольника, совпадающее с радиусом вписанной в первый многоугольник окружности. Построим чертеж:

\\begin{figure} [h!]
    \\centering
    \\renewcommand{\\figurename}{Рис. 6}
    \\renewcommand{\\thefigure}{}
    \\center{\\includegraphics[scale=0.35]{c_1.png}}
    \\caption{Чертеж для построения нулевой вершины и центра поворота $b$}
    \\label{fig:4_1.png}
\\end{figure}

\\par Применим к полученному прямоугольному треугольнику двойственную теорему косинусов: \\[cos(\\frac{\\pi}{2}) = -cos(\\frac{\\pi}{P})cos(\\frac{\\pi}{Q}) + sin(\\frac{\\pi}{P})sin(\\frac{\\pi}{Q})ch(\\rho_l),\\]
где $\\rho_l$ - расстояние в геометрии Лобачевского от центра круга до нулевой вершины.
Тогда:
\\[ch(\\rho_l) = \\frac{cos(\\frac{\\pi}{P})cos(\\frac{\\pi}{Q})}{sin(\\frac{\\pi}{P})sin(\\frac{\\pi}{Q})},\\]
\\[\\rho_l = arch(\\frac{cos(\\frac{\\pi}{P})cos(\\frac{\\pi}{Q})}{sin(\\frac{\\pi}{P})sin(\\frac{\\pi}{Q})})\\]
Исходя из того, что поворот относительно поворот относительно центра круга - движение и в геометри Лобачевского, и в Евклидовой геометрии, $\\rho_l$ будет равно расстоянию в геометри Лобачевского от центра круга до точки $(0, \\rho_e)$:
\\[\\rho_l = |ln[0, \\rho_e; -1, 1]|,\\]
\\[\\rho_l = |ln(\\frac{\\rho_e - 1}{0 - 1}:\\frac{\\rho_e + 1}{0 + 1})|,\\]
\\[\\rho_l = |ln(\\frac{1 - \\rho_e}{1 + \\rho_e})|,\\]
\\[\\frac{1 - \\rho_e}{1 + \\rho_e} = e^{-\\rho_l},\\]
\\[\\rho_e = \\frac{1 - e^{-\\rho_l}}{1 + e^{-\\rho_l}},\\]
\\[\\rho_e = \\frac{1 - e^{-arch(\\frac{cos(\\frac{\\pi}{P})cos(\\frac{\\pi}{Q})}{sin(\\frac{\\pi}{P})sin(\\frac{\\pi}{Q})})}}{1 + e^{-arch(\\frac{cos(\\frac{\\pi}{P})cos(\\frac{\\pi}{Q})}{sin(\\frac{\\pi}{P})sin(\\frac{\\pi}{Q})})}}\\]
Применим двойственную теорему косинусов второй раз:
\\[cos(\\frac{\\pi}{Q}) = -cos(\\frac{\\pi}{P})cos(\\frac{\\pi}{2}) + sin(\\frac{\\pi}{P})sin(\\frac{\\pi}{2})ch(h_l),\\]
где $h_l$ - расстояние в геометрии Лобачевского от центра круга до середины стороны первого многоугольника. Тогда:
\\[cos(\\frac{\\pi}{Q}) = sin(\\frac{\\pi}{P})ch(h_l),\\]
\\[ch(h_l) = \\frac{cos(\\frac{\\pi}{Q})}{sin(\\frac{\\pi}{P})},\\]
\\[h_l = arch(\\frac{cos(\\frac{\\pi}{Q})}{sin(\\frac{\\pi}{P})})\\]
И, аналогично тому, как мы выражали $\\rho_e$:
\\[h_e = \\frac{1 - e^{-h_l}}{1 + e^{-h_l}},\\]
\\[h_e = \\frac{1 - e^{-arch(\\frac{cos(\\frac{\\pi}{Q})}{sin(\\frac{\\pi}{P})})}}{1 + e^{-arch(\\frac{cos(\\frac{\\pi}{Q})}{sin(\\frac{\\pi}{P})})}},\\]
Мы получили выражение для $\\rho_e$ и $h_e$ через $P$ и $Q$, а значит, знаем координаты для нулевой вершины и центра поворота $b$.
\\subsubsection{Построение отрезка прямой}
Пусть есть точки $K$ и $M$ внутри круга. Рассмотрим 2 случая:
\\begin{enumerate}
    \\item \\textbf{Евклидова прямая $KM$ проходит через центр круга} \\\\
    В этом случае Евклидова прямая $KM$ в геометрии Лобачевского также будет прямой, поэтому построение тривиально.
    \\item \\textbf{Евклидова прямая $KM$ не проходит через центр круга} \\\\
    Сделаем инверсию точки $K$ относительно абсолюта, назовем ее образ $K'$. По свойству инверсии, окружность, проходящая через $K, M$ и $K'$ ортогональна абсолюту, поэтому ее дуга $KM$ в геометрии Лобачевского будет отрезком прямой. Таким образом задача сводится к построению дуги описанной окружности треугольника $KMK'$. Эта задача легко решается циркулем и линейкой. Кроме того, как минимум один из спосов ее решения в координатах очевиден: центр этой окружности будет решением системы из уравнений серединных перпендикуляров к $KM$ и $MK'$. При программировании же я использовал готовые формулы, найденный мной в сети. Координаты центра по ним вычисляются следующим образом:
    \\[x_c = \\frac{1}{D}((x_K^2 + y_K^2)(y_M - y_{K'}) + (x_M^2 + y_M^2)(y_{K'} - y_{K}) + (x_{K'}^2 + y_{K'}^2)(y_K - y_{M})),\\]
    \\[ y_c = \\frac{1}{D}((x_K^2 + y_K^2)(x_M - x_{K'}) + (x_M^2 + y_M^2)(x_{K'} - x_{K}) + (x_{K'}^2 + y_{K'}^2)(x_K - x_{M})),\\]
    где $D = 2(x_{K}(y_M - y_{K'}) + x_M(y_{K'} - y_K) + x_{K'}(y_{K} - y_{M}))$.
\\end{enumerate}
\\\\
\\par Итак, нами получены способы построения первого многоугольника и любого отрезка прямой в модели Пуанкаре в круге, реализованы преобразования, применяемые к этому многоугольнику и построен алгоритм, позволяющий при помощи этих преобразований построить нужное количество слоев без перекрытий.
\\section{Примеры замощений}
Здесь я приведу несколько, на мой взгляд, интересных примеров построенных замощений:
\\begin{figure} [h!]
    \\centering
    \\renewcommand{\\figurename}{Рис. 7}
    \\renewcommand{\\thefigure}{}
    \\center{\\includesvg{i_6_4.svg}}
    \\caption{$P=6$, $Q=4$, 4 слоя}
    \\label{fig:4_1.png}
\\end{figure}

\\begin{figure} [h!]
    \\centering
    \\renewcommand{\\figurename}{Рис. 8}
    \\renewcommand{\\thefigure}{}
    \\center{\\includegraphics[scale=0.36]{i_4_5.png}}
    \\caption{$P=4$, $Q=5$, 4 слоя}
    \\label{fig:4_1.png}
\\end{figure}

\\begin{figure} [h!]
    \\centering
    \\renewcommand{\\figurename}{Рис. 9}
    \\renewcommand{\\thefigure}{}
    \\center{\\includegraphics[scale=0.36]{i_3_12.png}}
    \\caption{$P=3$, $Q=12$, 3 слоя}
    \\label{fig:4_1.png}
\\end{figure}

\\begin{figure} [h!]
    \\centering
    \\renewcommand{\\figurename}{Рис. 10}
    \\renewcommand{\\thefigure}{}
    \\center{\\includegraphics[scale=0.36]{i_5_4.png}}
    \\caption{$P=5$, $Q=4$, 4 слоя}
    \\label{fig:4_1.png}
\\end{figure}

\\begin{figure} [h!]
    \\centering
    \\renewcommand{\\figurename}{Рис. 11}
    \\renewcommand{\\thefigure}{}
    \\center{\\includegraphics[scale=0.36]{i_7_3.png}}
    \\caption{$P=7$, $Q=3$, 5 слоев}
    \\label{fig:4_1.png}
\\end{figure}

\\begin{figure} [h!]
    \\centering
    \\renewcommand{\\figurename}{Рис. 12}
    \\renewcommand{\\thefigure}{}
    \\center{\\includegraphics[scale=0.36]{i_4_8.png}}
    \\caption{$P=4$, $Q=8$, 3 слоя}
    \\label{fig:4_1.png}
\\end{figure}

\\begin{figure} [h!]
    \\centering
    \\renewcommand{\\figurename}{Рис. 13}
    \\renewcommand{\\thefigure}{}
    \\center{\\includegraphics[scale=0.36]{i_10_3.png}}
    \\caption{$P=10$, $Q=3$, 4 слоя}
    \\label{fig:4_1.png}
\\end{figure}

\\end{document}